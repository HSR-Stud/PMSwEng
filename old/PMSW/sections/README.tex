\thispagestyle{empty}
\setcounter{page}{0} %Set PageNumber to 0
{\huge README }
\section*{Beschreibung}
Zusammenfassung für Projektmanagement und Software Engineering auf Grundlage der Vorlesung FS 16 von Hans Heinrich Pletscher auf der Vorlage von H.Badertscher \newline
Bei Korrekturen oder Ergänzungen wendet euch an einen der Mitwirkenden.

\section*{Modulschlussprüfung}
Kompletter Stoff aus Skript, Vorlesung, Übungen und Praktikum \newline
Schwerpunkt ist der in den Übungen behandelte Stoff, insbesondere Qt

\textbf{Die Prüfung besteht aus 2 Teilen:}\newline
% \usepackage{array} is required
\begin{tabular}{p{1.5cm} p{3cm} p{10cm}}
    \textbf{ 1.Teil}   & closed Book & Theoretische Fragen zum ganzen Prüfungsinhalt \\ 
    \textbf{ 2.Teil}   & semi-open book & Aufgaben im Stil der Übungen, Praktika und der in den Vorlesungen gelösten Aufgaben \\ 
\end{tabular} 

\subsection*{Plan und Lerninhalte}
{\scriptsize 
Werkzeuge und Techniken\newline
    \begin{itemize}
        \item Versionsverwaltung mit Subversion
        \item Unit-Testing mit CPPUnit
        \item Generierung der Dokumentation aus dem Source-Code mit Hilfe von Doxygen
        \item Erstellen von GUI-Programmen mit Hilfe der qt-Library.
    \end{itemize}
Software Entwicklung
\begin{itemize}
    \item Vorgehensmodelle
    \item Software Projektmanagement
    \item Testen von Software (u.a. Unit-Testing)
    \item Refactoring (Überarbeitung, Verbesserung bestehender Software)
    \item Allgemeine Entwurfsprinzipien: Design by Contract, defensives Programmieren
    \item Ereignisbasierte Programmierung, Entwurf von GUI-Programmen
\end{itemize}
}
\vfill
\section*{Contributors}
\begin{tabular}{ll}
    Michel Gisler & michel.gisler@hsr.ch \\
    Stefan Reinli & stefan.reinli@hsr.ch \\ 
    Luca Mazzoleni& luca.mazzoleni@hsr.ch \\ 
    Hannes Badertscher& hannes.badertscher@hsr.ch \\    
\end{tabular} 

{\scriptsize 
    \section*{License}
    \textbf{Creative Commons BY-NC-SA 3.0}
    
    Sie dürfen:
    \begin{itemize}
        \item Das Werk bzw. den Inhalt vervielfältigen, verbreiten und öffentlich
        zugänglich machen.
        \item Abwandlungen und Bearbeitungen des Werkes bzw. Inhaltes anfertigen.
    \end{itemize}
    Zu den folgenden Bedingungen:
    \begin{itemize}
        \item Namensnennung: Sie müssen den Namen des Autors/Rechteinhabers in der von ihm
        festgelegten Weise nennen.
        \item Keine kommerzielle Nutzung: Dieses Werk bzw. dieser Inhalt darf nicht für
        kommerzielle Zwecke verwendet werden.
        \item  Weitergabe unter gleichen Bedingungen: Wenn Sie das lizenzierte Werk bzw. den
        lizenzierten Inhalt bearbeiten oder in anderer Weise erkennbar als Grundlage
        für eigenes Schaffen verwenden, dürfen Sie die daraufhin neu entstandenen
        Werke bzw. Inhalte nur unter Verwendung von Lizenzbedingungen weitergeben,
        die mit denen dieses Lizenzvertrages identisch oder vergleichbar sind.
    \end{itemize}
    Weitere Details: http://creativecommons.org/licenses/by-nc-sa/3.0/ch/
}
%If we meet some day, 
%and you think this stuff is worth it, you can buy me a beer in return.
\clearpage
\pagenumbering{arabic}% Arabic page numbers (and reset to 1)
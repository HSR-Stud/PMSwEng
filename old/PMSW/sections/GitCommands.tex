\section{Git Commands}
\begin{longtable}{| p{.30\textwidth} | p{.70\textwidth} |}
    \hline 
    \textbf{Create}&
    \\ \hline
    
    git init <directory>&
    Create empty Git repo in specified directory.\newline
    Run with no arguments to initialize the current directory as a git repository.
    \\ \hline 
    
    git clone <repo>& 
    Clone repo located at <repo> onto local machine.\newline
    Original repo can be located on the local filesystem or on a remote machine
    \\ \hline  \hline
    
    \textbf{Local Changes}&
    \\ \hline 

    \hline 
    git status&
    List which files are staged, unstaged, and untracked  
    \\ \hline
    
    git diff& 
    Show unstaged changes between your index and working
    directory 
    \\ \hline 
    
    git diff HEAD&
    Show difference between working directory and last commit.
    \\ \hline
        
    git add <directory>&
    Stage all changes in <directory> for the next commit. Replace <directory>
    with a <file> to change a specific file or witch <.> to Stage all.
    \\ \hline 
    
    git commit -m " \" <message>"&
    Commit the staged snapshot, but instead of launching a text editor, use
    <message> as the commit message.
    \\ \hline
    
    git commit -a&
    Commit all local changes in tracked files.  
    \\ \hline 
    
    gitt commit --amend&
    Change the last commit. \textbf{Don‘t amend published commits!}
    \\ \hline \hline
    
    \textbf{Commit History}&  
    \\ \hline 

    \hline
    git log&
    Display the entire commit history using the default format.
    \\ \hline 
    
    git log -p <file>&
    Show changes over time for a specific file 
    \\ \hline
    
    git log --author="\"<pattern>"&
    Search for commits by a particular author. 
    \\ \hline
        
    git blame <file>& 
    Who changed what and when in <file>
    \\ \hline \hline

   \textbf{Branches and Tags}&  
    \\ \hline

    \hline
    git branch&
    List all of the branches in your repo. Add a <branch> argument to
    create a new branch with the name <branch>.
    \\ \hline 
    
    git branch <new-branch>&
    Create a new branch based
    on your current HEAD
    \\ \hline
    
    git branch -d <branch>&
    Delete a local branch
    \\ \hline   
    
    git branch -dr <remote/branch>&
    Delete a branch on the remote
    \\ \hline
        
    git checkout -b <branch>&
    Create and check out a new branch named <branch>. Drop the -b
    flag to checkout an existing branch.
    \\ \hline 
    
    git merge <branch>&
    Merge <branch> into the current branch.
    \\ \hline
    
    git tag <tag-name>&
    Mark the current commit with a tag (v1.0.1)
    \\ \hline  \hline
      
    \textbf{Update and Publish}&
    \\\hline 

    \hline
    git remote -v&
    List all currently configured remotes.
    \\ \hline 
    
    git remote add <name> <url>& 
    Add new remote repository, named <remote>
    \\ \hline 
    
    git fetch <remote>&  
    Download all changes from <remote>,
    but don‘t integrate into HEAD
    \\ \hline
    
    git pull <remote> <branch>&
    Download changes and directly merge/integrate into HEAD  
    \\ \hline 
    
    git push <remote> <branch>&
    Publish local changes on a remote
    \\ \hline \hline
       
   \textbf{ Merge and Rebase}&
    \\\hline 
    
    \hline
    git merge <branch>&
    Merge <branch> into your current HEAD
    \\ \hline 
    
    git rebase <branch>& 
    Rebase your current HEAD onto <branch>!\textbf{Don‘t rebase published commits!}
    \\ \hline 
    
    git rebase --abort&
    Abort a rebase
    \\ \hline 
    
    git mergetool& 
    Use your configured merge tool to solve conflicts
    \\ \hline \hline
   
    \textbf{Undo}&
    \\\hline 

    \hline
    git reset --hard HEAD&
    Discard all local changes in your working directory.
    \\ \hline  
    
    git checkout HEAD <file>&
    Discard local changes in a specific file
    \\ \hline   
    
    git revert <commit>&
    Revert a commit (by producing a new commit with contrary changes)
    \\ \hline    
    
    git reset <commit>&
    Move the current branch tip backward to <commit>, reset the staging area to match, but leave the working directory alone. 
    \\ \hline
\end{longtable}

\subsection{Bashprints}
\begin{multicols}{2}
\begin{minipage}[l]{.40\textwidth}
    Files changed but not staged.\\    
    \textbf{\$ git status}\\
    On branch master\\
    Your branch is up-to-date with 'origin/master'.\\
    Changes not staged for commit:\\
    \quad (use "git add <file>..." to update what will be committed)\\
    \quad (use "git checkout -- <file>..." to discard changes in working directory)\\
    \\
    \qquad modified:\quad idiotenseite/IdiotenseiteInclude.tex\\
    \\
    no changes added to commit (use "git add" and/or "git commit -a")\\
\end{minipage}

\begin{minipage}[r]{.40\textwidth}
    \textbf{\$ git log}\\
    commit 1b7ff854a491397087ff89689a535c275549e6ee\\
    Author: \quad Michel Gisler <Michel Gisler>\\
    Date:  \quad  Tue May 31 15:08:58 2016 +0200\\
    \\  
    \qquad Kleiner Ergänzungen und Anpasungen\\
\end{minipage}
\end{multicols}

\clearpage
\chapter{Lernziele}

‐ Unterschied zwischen Library und Framework kennen und begründen können \\
‐ Unterschied zwischen statischer und dynamischer Library kennen und begründen können \\
‐ Statische Libraries erstellen und mit ausführbaren Anwendungen linken können \\

‐ Grundlagen von Qt kennen und anwenden können: \\
‐ moc \\
‐ Qobject \\
‐ qmake \\
‐ Qapplikation \\

‐ mit QWidgets und deren abgeleiteten Klassen GUI erstellen können\\
‐ GUIs mit Hilfe von GUI Design erstellen können \\
Selbststudium: \\
‐ Die in der Widget‐Kollektion gezeigt Widget sind zur Kenntnis zu nehmen \\

‐ Eventbasiertes Programmieren verstehen und anwenden können \\
‐ Unterschied zwischen Polling und Eventsbasierten Programmieren kennen \\
‐ Qt Signals/Slots Prinzip verstehen und anwenden können, indem eigene Signals und Slots miteinander verbunden werden können. \\

‐ GUI + Interaktion + Problem Domain anwenden \\
‐ Ziele des Projektmanagments in Softwareengineering kennen \\
‐ Projektphasen in Zusammenhang mit dem Softwarelebenszyklus kennen und einordnen können \\
Selbststudium: \\
‐ Use‐Case Diagramme aufgrund vorgegebener Use‐Cases und Aktoren erstellen können. \\
‐ Wissen, was Aktoren, Systemgrenze und Use‐Cases sind \\

‐ Verbreitete Software Vorgehensmodelle mit deren Eigenschaften kennen, damit Vor‐ und Nachteile der verschiedenen Modelle daraus gezogen werden 
können. 

‐ Aufwandschätzungsarten mit deren Eigenschaften kennen, damit bei einer Anwendung die beste Methode gewählt werden kann. \\
Selbstudium: \\
‐ Wissen, wie das Aron‐Model und COCOMO funktioniert \\
‐ COCOMO II anwenden können, indem ein Aufwand basierend auf Function Points mit COCOMO II in Early‐Design geschätzt werden kann \\

‐ Projektstrukturpläne kennen \\
‐ Arbeitspakete erstellen können \\
‐ Netzplantechnik und bekannte Methoden davon kennen \\
‐ Kritischer Pfad in einem Netzplan finden können \\
‐ Gantt‐Chart mit MS Project erstellen können \\

‐ Pflichtenheft mit IEEE Schablone erstellen können 

‐ Verschiedene Versionsverwaltungskonzepte verstehen \\
‐ Git Basics kennen und anwenden können. Dabei sollen die Grundbefehle ohne Unterlagen angewendet werden können.                    \\                                            
‐ Git Branch Verwaltung verstehen und mit Unterlagen anwenden können. Dazu gehört auch das mergen von Branches. \\                                                      
‐ Workflows mit entferntes Repositories anwenden können \\
Selbststudium: \\
‐ Verstehen wie Git Objekte anlegt und speichert. Dazu gehören auch trees und blobs. \\
‐ Stashing verstehen und in einer geeigneten Situation anwenden können \\
‐ Wissen, dass es einen Credential Storage gibt \\
‐ Unterschied zwischen Checkout/Reset auf Branches oder Files kennen und mit Unterlagen anwenden können

Selbststudium: \\
‐ Einfacher Workflow durchspielen, verstehen und anwenden können \\
‐ Wichtigste Markdown Befehle anwenden können \\
‐ Markdown File als Mainpage in Doxygen nutzen können \\
‐ Weitere nützliche Befehle und Styles von Doxygen kennen

‐ Unit‐Test schreiben können \\
‐ Testfälle anhand Aquivalenzklassen mit Grenzwertanalyse erstellen \\
‐ Zwischen Anweisungs‐, Zweig‐, und Pfadüberdeckung unterscheiden können und die jeweiligen Überdeckungen anwenden können \\
‐ Greybox‐Test kennen und anwenden können

‐ Code Coverage mit gcov anwenden können \\
‐ Unit‐ und Integrationstests googletest umsetzten können \\
‐ Anwendungsgebiet von QtTest kennen und anwenden können \\
‐ Klassen mit Dependency injection schreiben, damit Unit‐Test überhaupt möglich ist

‐ Design by Contract Prinzip verstehen und anwenden können zwischen Funktionsaufrufer und Funktion \\
‐ Defensives Programmieren im Zusammenhang mit DbC verstehen. Wann können asserts angewendet werden, wann ist defensives Programmieren mit 
Exception oder return Codes angebracht. \\
‐ Begriff Refactoring kennen \\
‐ Einige einfache Refactoring‐Methoden kennen und anwenden können
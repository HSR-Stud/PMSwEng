\section{Versionskontrolle}
\subsection{Versionskontrollsysteme (VCS)}

\begin{multicols}{2}
	\subsubsection{Zweck}
	\begin{itemize}
		\item Aufbewahrung, Verwaltung, Wiederherstellen von Dateien \newline in einem Archiv
		\item Koordination des Zugriffs
	\end{itemize}
	
	\subsubsection{Einsatz}
	\begin{itemize}
		\item Softwareentwicklung
		\item Content-Management-Systeme (CMS)
		\item Archiv $=$ Repository $=$ Lager
	\end{itemize}
\end{multicols}

\subsubsection{Einteilung}
	\begin{itemize}
		\item Lokale Versionskontrollsysteme (LVCS), ein Archiv für jede Datei
		\item Zentralisierte Versionskontrollsysteme (CVCS), ein Archiv für alles
		\item Verteilte Versionskontrollsysteme (DVCS), mehrere verteilte Archive
\end{itemize}

\subsection{Git}
\begin{minipage}{7cm}
\begin{itemize}
	\item Git $\rightarrow$ Blödmann
	\item Entwicklet von Linus Torvalds, Linux-Erfinder
	\item Man arbeitet mittels einer Konsole, kein GUI
	\item Git ist ein DVCS
\end{itemize}
\end{minipage}
%\hfill
\begin{minipage}{9cm}
	\includegraphics[width=11cm]{images/git.png}
\end{minipage}

\subsubsection{Allgemeines}
\begin{tabular}{|l|l|}
	\hline \textbf{Repository} &
    Ist ein Archiv für ein Projekt und enthält alle Änderungen und Versionen
    \\ 	\hline     
    \textbf{Branch} &
    Beschreibt zusammenhängende Änderungen in einem Projekt. Es gibt Minimum einen  bis beliebig viele.
    \\
    &
    Der Master-Branch ist der Produktivzweig
    \\ \hline
    \textbf{Commit} &
    Beschreibt eine Änderung in einem Branch an einer Datei mit Änderungsinformation
    \\ \hline
    \textbf{Snapshot} &
    Momentanes Zeitabbild des Projektes
    \\ \hline
    \textbf{Merge} &
    Zusammenführen von Änderungen aus zwei Branches
    \\	\hline
    \textbf{Stagen} &
    Datei zum Index hinzufügen
    \\\hline
\end{tabular}
\\
\\
\begin{tabular}{|c|c|}
	\hline \textbf{Lifecycle von Dateien} & \textbf{Branch}\\
	\hline \tabbild[width=9cm]{images/git_lifecycle.png} & \tabbild[width=9cm]{images/git_branch.png}\\
	\hline
\end{tabular}

\subsubsection{Datenbereiche}
\begin{tabular}{|l|l|}
	\hline
	\begin{tabular}[c]{@{}l@{}}\textbf{Workspace}\\ -Projektverzeichniss\\ -hier wird gearbeitet\end{tabular}   & \begin{tabular}[c]{@{}l@{}}\textbf{Repository}\\ -Versionsgeschichte in Form von Commits (Revision)\\ -lokal/remote\end{tabular}   \\ \hline
	\begin{tabular}[c]{@{}l@{}}\textbf{Stash}\\ -Lager\\ -temporäres Speichern der Workspace-Daten\end{tabular} & \begin{tabular}[c]{@{}l@{}}\textbf{Staging Area/Index/Cache}\\ -sammelt Änderungen für Commits\\ -Bereitstellungsraum\end{tabular} \\ \hline
\end{tabular}

\subsubsection{Git-Konzepte}
\begin{itemize}
	\item Git-Workspace ist ein Verzeichnis welches die zu bearbeitenden Dateien eines Projektes enthält
	\item In diesem Verzeichnis ist ein Unterverzeichnis \textit{".git"}
	\item Dieses Unterverzeichnis \textit{".git"} bildet das lokale Repository
	\item Im Branch zeigen die Pointer immer auf den vorhergehenden Commit
	\item Jeder Commit bekommt einen speziellen Hashwert
\end{itemize}
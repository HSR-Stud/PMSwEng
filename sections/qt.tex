\section{Qt}
\begin{minipage}{17cm}
	Qt (sprich cute =süess) ist eine plattformübergreifende Programmierungsapplikation für grafische Benutzeroberflächen (GUI = Graphical User Interface).
\end{minipage}
\begin{minipage}{1.5cm}
	\includegraphics[width=1.5cm]{images/qt_logo.jpg}
\end{minipage}

\subsection{Grundlagen zu Qt}
\begin{multicols}{2}
\subsubsection{Qt Widget}
	\begin{itemize}
		\item Qt C++ Klassenbibliothek stellt GUI-Elemente (Widget) zur Verfügung
		\item Das GUI wird als C++ Sourcefile geschrieben
	\end{itemize}
\subsubsection{Qt Quick}
	\begin{itemize}
		\item basiert auf QML , Sprache welche das Aussehen festlegt.
		\item ähnlich wie HTML
	\end{itemize}	
\end{multicols}

\subsubsection{Qt Konvention}
\begin{tabular}{|l|l|l|}
	\hline \textbf{Was} & \textbf{Beispiel} &\textbf{Konvention}\\
	\hline Qt-Modulname & "QtCore"& \textit{Qt}\\
	\hline Qt-Klassenname & "QString" & \textit{Q}\\
	\hline Qt-Variablen-,Funktionsname & "qTranslator, qDebug()"& \textit{q}\\
	\hline Qt-include & "\#include <QtGui>& \textit{ohne.h}\\
	\hline
\end{tabular}

\subsubsection{Qt Building}
\begin{itemize}
	\item qmake Makefile-Generator
	\item macht aus plattformunabhängiem Projektfile (.pro) ein plattformspezifisches Projektfile. 
	\item die gewünschte Plattform wird qmake als Option mitgegeben
	\item Bei Build-Problemen alle Dateien ausser .pro und Sourcefile löschen
\end{itemize}

\subsubsection{Qt Widget}
\begin{itemize}
	\item \textit{"Widget $\rightarrow$ Window Gadget / Fensterding"}
	\item Visuelles Element einer UI
	\item Toplevel Widget, alleinstehendes Fenster 
	\item Verschachtelung, in einem Widget sind weitere Widget enthalten
\end{itemize}

\subsection{GUI-Programmierung}
\subsubsection{Layout}
Festlegen Anordnung und Grösse der Widgets in einem Toplevel Widget.
\begin{enumerate}
	\item Absolute Positionierung\\
	Mit Methoden wie \textit{"'setGeometry(x,y,w,h)"}	
	\item Layout Manager\\
	\begin{tabular}{|l|l|}
		\hline \textbf{Name} & \textbf{Ort}\\
		\hline QVBoxLayout & Elemente vertikal\\
		\hline QHBoxLayout & Elemente horizontal\\
		\hline QGridLayout & zweidimensionales Gitter\\
		\hline QFormLayout & Elemente zeilenweise\\
		\hline QStackedLayout & aufeinandergelegt\\
		\hline
	\end{tabular}
	\item GUI-Designer\\
	Anordnung innerhalb Formular, anschliessend Umwandlung in Code	
\end{enumerate}

\subsubsection{Interaktion}
Reaktion des Programmes auf eine Eingabe.
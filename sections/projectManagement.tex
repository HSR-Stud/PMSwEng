\section{Projektmanagement}
\subsection{Grundlagen}

\subsubsection{Definition eines Projektes}
Ein Projekt ist ein zeitlich beschränktes Vorhaben zur Erzeugung eines einmaligen Produkt oder Dienst. 

\subsubsection{Definition des Projektmanagement}
Das Projektmanagement bezeichnet die Gesamtheit von Führungsaufgaben, -organisation, -techniken und -mitteln für die Initiierung, Definition, Planung, Steuerung und den Abschluss von Projekten. 

\begin{multicols}{2}
	\subsubsection{Merkmale eines Projektes}
	\begin{itemize}
		\item einmaliges Vorhaben
		\item klare Ziele
		\item hat Risiken
		\item zeitlich begrenzt
		\item Kostenrahmen
		\item hat innovativen Charakter
		\item es arbeiten mehrere Personen daran
		\item hat Projektleiter \& Projektteam 
	\end{itemize} 
	
	\subsubsection{Kategorien von Projekten}
	\begin{itemize}
		\item Forschungs- und Entwicklungsprojekte
		\begin{itemize}
			\item  Entwicklung neuer Produkte
			\item  Erstellung von Software
		\end{itemize}			 
		\item Investitionsprojekte
		\begin{itemize}
			\item Bau eines Flughafen			
		\end{itemize}
		\item Organisationsprojekte
		\begin{itemize}
			\item Einführung Software
			\item Vorbereitung Veranstaltung
		\end{itemize}
	\end{itemize}
\end{multicols}

\subsubsection{Die 3 Eckpfeiler eines Projektes}
\begin{minipage}{10cm}
	Diese drei Grössen sind voneinander abhängig und sind auch bekannt als \textit{"Fast, Good, Cheap"}
	\begin{itemize}
		\item Ergebnis \textit{"Scope"}
		\item Zeit \textit{"Time"}
		\item Aufwand \textit{"Cost"}
	\end{itemize}
\end{minipage}
\begin{minipage}{3cm}
	\includegraphics[width=3cm]{images/eckpfeiler.png}
\end{minipage}
\begin{minipage}{3cm}
	\includegraphics[width=3cm]{images/dreieck.png}
\end{minipage}

\subsubsection{Das Projektdreieck (Magisches Dreieck)}
\begin{minipage}{15cm}
	Diese drei Grössen werden als Dreieck dargestellt. Die Aufgabe des Projektmanagement ist es, ein sinnvolles Verhältnis dieser 3 herzustellen und während der Projektdauer zu gewährleisten. 
\end{minipage}

\subsubsection{Projektbeteiligten (Stakeholder)}
\begin{itemize}
	\item Auftraggeber $\rightarrow$ bezahlt \& befiehlt
	\item Projektleiter (PL) $\rightarrow$ Manager, weniger Fachexperte
	\item Projektmitarbeiter (Projektteam) $\rightarrow$ Fachexperten, hohe Methoden und Sozialkompetenz
\end{itemize}

\subsubsection{Projektorganisationsformen}
\begin{multicols}{3}
	\begin{itemize}
		\item \textbf{Reine Projektorganisation}
		\begin{itemize}
			\item Projektleiter \& Projektteam arbeiten Vollzeit am Projekt
			\item Mitarbeiter sind vollkommen Projektleiter unterstellt
			\item sehr selten
		\end{itemize}
		\item \textbf{Matrix-Projektorganisation}
		\begin{itemize}
			\item Mitarbeiter arbeiten Teilzeit am Projekt
			\item Verantwortung ist aufgeteilt zwischen Projektleiter/Linieninstanzen
			\item Häufigste Form
		\end{itemize}
		\item \textbf{Stabs-Projektorganisation}
		\begin{itemize}
			\item Hierarchie ist unverändert
			\item Projektkoordinator hat keine Weisungsbefugnisse
			\item stimmt Zusammenarbeit mit Mitarbeiter ab
		\end{itemize}
	\end{itemize}
\end{multicols}

\subsection{Projektablauf}
Jedes Projekt hat ein Anfang und ein Ende (Lebenszyklus). Der vorzeitige Abbruch ist ein Unfall, sogennantes ungeplantes Ende. Jedes Projekt besteht aus 4 Phasen. Die 4 Phasen überlappen sich und laufen parallel ab. Während der Durchführung ist auch die Planung anzupassen. \\
\\
\begin{minipage}{6cm}
	\textbf{Projektphasen:}
	\begin{enumerate}
		\item Projektdefinition
		\item Projektplanung
		\item Projektdurchführung
		\item Projektabschluss
	\end{enumerate}
\end{minipage}
\begin{minipage}{7cm}
	\includegraphics[width=8cm]{images/projektablauf.png}
\end{minipage}

\subsection{Projektphasen}
\subsubsection{Projektdefinition (Vorphase, Initiierung)}
\begin{itemize}
	\item Vorbereiten des Projektes
	\item Vor dem eigentlichen Projekt
	\item Ziele festlegen (Das Wichtigste eines Projektes)
	\item Termine, Kosten und Projektorganistation festlegen
	\item Wird in Projektauftrag niedergeschrieben
\end{itemize}
	\begin{minipage}{11cm}
		\textbf{Projektauftrag} \\
		Der Projektauftrag ist ein schriftliches Dokument wo Ziele und Rahmenbedingungen festgehalten sind. Das Dokument wird vom Auftraggeber unterzeichnet, damit ist Existenz des Projektes formell bestätigt. Die Ziele des Projektes sollten gemäss SMART formuliert sein. 
	\end{minipage}
	\begin{minipage}{6cm}
		\includegraphics[width=6cm]{images/smart.png}
	\end{minipage}
	
\subsubsection{Projektplanung}
\begin{minipage}{11cm}
	\textbf{1. Projektstrukturplan (PSP)}
	\begin{itemize}
		\item Projekt in Teilprojekte zerlegen
		\item Kleinstmögliche Aufgaben sind Arbeitspakete
		\item Arbeitspakete sind phasenorientiert oder objektorientiert oder funktionsorientiert aufgeteilt
		\item Pro Arbeitspaket wird eine Arbeitsbeschreibung erstellt
	\end{itemize} 	
\end{minipage}
\begin{minipage}{7cm}
	\includegraphics[width=7cm]{images/projektstrukturplan.png}
\end{minipage}
\\
\\
\\
\textbf{2. Projektablaufplan (PAP)} \\
Stellt die logische Abhängigkeiten der Arbeitspakete dar. Dazu wird eine Vorgangsliste erstellt, welche die Abhängigkeiten zwischen den Arbeitspakete aufzeigt. 
\renewcommand{\arraystretch}{1.2}
\begin{table}[h!]
	\begin{tabular}{|l|l|l|}
		\hline \textbf{Abhängigkeit} & \textbf{Funktion} & \textbf{Bild} \\
		\hline Normalfolge & Nachfolger kann erst beginnen, wenn Vorgänger beendet ist & \tabbild[width=4cm]{images/normalfolge.png}\\
		\hline Anfangsfolge & Nachfolger kann erst beginnen, wenn Vorgänger begonnen hat &
		\tabbild[width=4cm]{images/anfangsfolge} \\
		\hline Endfolge & Nachfolger kann erst enden, wenn Vorgänger beendet ist &
		\tabbild[width=4cm]{images/endfolge.png} \\
		\hline Sprungfolge & Nachfolger kann erst enden wenn Vorgänger begonnen hat &
		\tabbild[width=4cm]{images/sprungfolge.png}\\
		\hline
	\end{tabular}
\end{table}

\pagebreak
\textbf{3. Projektterminplan}
\begin{itemize}
	\item Zuordnung von Terminen zu Arbeitspaketen
	\item Festlegen der Meilensteine
    \begin{itemize}
    	\item Meilensteine sind Etappenziele welche zur Projektkontrolle dienen
    	\item Meilensteine sind so formuliert, dass \textit{"'Erfüllt"} oder \textit{" Nicht Erfüllt"} gilt
    \end{itemize}
	\item Projektterminplan wird aufgrund des PSP und PAP erstellt
\end{itemize}
\begin{table}[h!]
	\begin{tabular}{l l l}
		 \textbf{Balkendiagramm} & \textbf{Netzplan} & \textbf{Microsoft Project} \\
		 \includegraphics[width=6cm]{images/balkendiagramm} & \includegraphics[width=6cm]{images/netzplan}	& \includegraphics[width=6cm]{images/msproject}	 
	\end{tabular}
\end{table}
\vspace{0.2cm}
\textbf{4. Ressourcen- und Kapazitätsplan}\\
Die benötigten Ressourcen ermitteln und mit den Kapazitäten abstimmen. 
\\
\\
\textbf{5. Kosten- und Budgetplan}\\
Die Kosten für für die Ressourcen schätzen und Budget erstellen. 

\subsubsection{Projektdurchführung}
Die eigentliche Arbeit beginnt. Der Projektleiter hat die Aufgabe des Projekt-Controlling (to control $\rightarrow$ steuern), dieses besteht aus der Projektkontrolle und der Projektsteuerung. 
\renewcommand{\arraystretch}{1.2}
\begin{table}[h!]
	\begin{tabular}{|l|l|}
		\hline \textbf{Projektkontrolle} 	& Rechtzeitiges Feststellen von Abweichungen gegenüber dem geplanten.  \\ 
		& Soll-Ist-Vergleich\\
		\hline  \textbf{Projektsteuerung} 	& Massnahmen um Projekt bei Abweichungen wieder auf Zielkurs zu bringen. \\
		& Soll-Werte anpassen\\
		\hline
	\end{tabular}
\end{table}
\subsubsection{Projektabschluss}
\begin{itemize}
	\item Projektarbeit erfolgreich abgeschlossen
	\item Ziele sind erreicht
	\item Formeller Abschluss mit Auftraggeber
	\item Projektteam interner Abschluss
\end{itemize}
